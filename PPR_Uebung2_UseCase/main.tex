\documentclass[a3paper,11pt]{article}
\usepackage{tikz}
\usepackage{tikz-uml}

\begin{document}

\section*{Use-Case-Diagramm}
\resizebox{\textwidth}{!}{
\begin{tikzpicture}
% Akteure
\umlactor[x=-10, y=8]{Admin}
\umlactor[x=-10, y=0]{Gäste}

% Anwendungsfälle
\umlusecase[x= -2, y=10]{Alle Abgeordnete erfassen}
\umlusecase[x= -2,y=8]{Fraktion/Gruppe von Bundestag ermitteln}
\umlusecase[x= -2,y=6]{Abgeordnete zuordnen}
\umlusecase[x= -2,y=4]{Sitzungen erfassen}
\umlusecase[x=2, y=2]{Kommentare und Zwischenrufe erfassen}
\umlusecase[x=5, y=0]{Alle Daten in einer Datenbank speichern}
\umlusecase[x=7, y=-2]{Die statische HTML-Seiten erzeugen}
\umlusecase[x=6, y=6.5]{Tagesordnung}
\umlusecase[x=6, y=4.5]{Rede der einzelnen Redner}

\umlassoc{Admin}{Alle Abgeordnete erfassen}
\umlassoc{Admin}{Fraktion/Gruppe von Bundestag ermitteln}
\umlassoc{Admin}{Abgeordnete zuordnen}
\umlassoc{Admin}{Sitzungen erfassen}
\umlassoc{Admin}{Kommentare und Zwischenrufe erfassen}
\umlassoc{Admin}{Alle Daten in einer Datenbank speichern}
\umlassoc{Admin}{Die statische HTML-Seiten aus Datenbank erzeugen}

\umlassoc{Gäste}{Tagesordnung}
\umlassoc{Gäste}{Rede der einzelnen Redner}

\umlinclude{Sitzungen erfassen}{Tagesordnung}
\umlinclude{Sitzungen erfassen}{Rede der einzelnen Redner}

\end{tikzpicture}
}
\end{document}
